\documentclass{article}
\input{preamble.tex}

\addbibresource{mybib.bib}

\begin{document}

\title{Derived categories of quasi-coherent sheaves}

\author{Ken Lee}
\date{March 2025}
\maketitle

This is an exposition to
define the unbounded derived stable $\infty$-category of quasi-coherent sheaves
on a scheme using stable $\infty$-categories and
prove that derived global sections can be computed by Cech cohomology,
hopefully in an obvious way!

\section{Derived stable $\infty$-categories for the working mathematician}

The goal of this section is to answer :
\begin{center}
  \text{Question : What should derived categories and derived functors be?}
\end{center}
% \begin{prop}

%   Let $A$ be an additive category.
%   Consider $N_{dg}(\CH A) \in \SSET$ the dg nerve of the differential graded
%   category of chain complexes in $A$.
%   Then $N_{dg}(\CH A)$ is a stable infinity category.
% \end{prop}
% \begin{proof}
%   \cite[Prop. 1.3.2.10]{lurie-HA}
% \end{proof}


% \begin{prop}
%   Let $A$ be Grothendieck abelian,
%   i.e. presentable abelian category
%   \footnote{
%     i.e. $A$ admits small colimits and there exists
%     a set of $\ka$-compact objects generating $A$ under small colimits
%     for some regular cardinal $\ka$.
%   }
%   such that filtered colimits are exact.
%   Let $\cD(A)$ be the full subcategory of $N_{dg}(\CH \, A)$
%   of $W$-local objects.\cite[Def. 5.5.4.1]{lurie-HTT}
%   Then \begin{enumerate}
%     \item $\cD(A)$ is presentable. 
%     \item there exists a left adjoint $L : N_{dg}(\CH \, A) \to \cD(A)$
%     to the inclusion.
%     \item $\cD(A)$ is stable.
%     \item 
%     Let $\CH_f A$ denote the full subcategory of $\CH\,A$
%     consisting of fibrant objects.
%     Then the inclusion $N_{dg}(\CH_f A) \subs N_{dg}(\CH\,A)$ 
%     induces an equivalence $N_{dg}(\CH_f A) \simeq \cD(A)$.
%     In other words, this definition agrees with 
%     \cite[Def. 1.3.5.8]{lurie-HA}
%     \item For a morphism $f$ in $N_{dg}(\CH\,A)$ with cofiber $Q$,
%     the following are equivalent : 
%     \begin{enumerate}
%       \item $f \in W$
%       \item $Q$ is acyclic
%       \item $LQ = 0$
%       \item $L(f)$ is an isomorphism in $\cD(A)$.
%     \end{enumerate}
%     Hence, the functor $L$ exhibits $\cD(A)$ as a
%     derived infinity category of $A$.
%     \item (Classical definition) 
%     Consider the full subcategory $\mathrm{Acyc}\,A \subs \CH A$
%     of acyclic complexes.
%     Then for all stable infinity categories $\cE$,
%     restriction along $L$ induces 
%     a fully faithful embedding of infinity categories \[
%       \FUN_{\mathrm{ex}}(\cD(A), \cE) \map{}{}
%       \FUN_{\mathrm{ex}}(N_{dg}(\CH  A) , \cE)
%     \]
%     where $\FUN_{\mathrm{ex}}$ means the infinity category of exact functors,
%     and the essential image consists of 
%     functors where every $X \in \mathrm{Acyc}\,A$ is mapped to a zero object.
%   \end{enumerate} 
% \end{prop}
% \begin{proof}
%   (1) and (2) are parts (2) and (3) of \cite[Prop. 5.5.4.15]{lurie-HTT}.

%   (3) \cite[Lem. 1.1.3.3]{lurie-HA}
  
%   (4) \cite[Prop. 1.3.5.14]{lurie-HA} shows that
%   $N_{dg}(\CH_f A) \subs \cD(A)$.
%   It remains to show essentially surjective.
%   Let $X \in \cD(A)$. We can choose a trivial cofibration $f : X \to Q$
%   where $Q \in N_{dg}(\CH_f A)$.
%   But by Yoneda and \cite[Prop. 1.3.5.14]{lurie-HA} again,
%   it follows that $f$ is an isomorphism in $\cD(A)$.

%   (5) \cite[Prop. 5.2.7.12]{lurie-HTT} gives the ``Hence''.
%   Now let $f : X \to Y$ and $Y \to Q$ the cofiber.
%   We have a commuting triangle \begin{cd}
%   	Q & LQ \\
% 	{\widetilde{Q}}
% 	\arrow[from=1-1, to=1-2]
% 	\arrow[from=1-1, to=2-1]
% 	\arrow[dashed, from=1-2, to=2-1]
%   \end{cd}
%   where $Q \to \TILDE{Q}$ is a trivial cofibration with $\TILDE{Q}$ fibrant,
%   the unit of the adjunction $Q \to LQ$ is an $W$-equivalence 
%   as in the proof of \cite[Prop. 5.5.4.15]{lurie-HA},
%   and $LQ \to \TILDE{Q}$ is due to the adjunction.
%   By two-out-of-three for $W$-equivalences,
%   $LQ \to \TILDE{Q}$ is an $W$-equivalence and hence an isomorphism.
%   Now $f \in W$ iff $Q$ is acyclic
%   iff $\TILDE{Q}$ is acyclic iff 
%   $0 \to \TILDE{Q} \in W$ iff
%   $0 = \TILDE{Q}$ iff $0 = LQ$ iff 
%   $L(f)$ is an isomorphism.

%   (6) $L : N_{dg}(\CH\,A) \to \cD(A)$ being right exact implies
%   it is exact.
%   Composition of exact functors between stable infinity categories
%   is again exact.
%   Fully faithfulness follows from (5).
%   Let $F : N_{dg}(\CH\,A) \to \cE$ be an exact functor.
%   If $F$ inverts quasi-isomorphisms,
%   then $F(Q) = 0$ for any acyclic $Q$ because $0 \to Q \in W$.
%   Conversely, if $F$ kills acyclics,
%   then $f \in W$ implies its cofiber $Q$ is acyclic,
%   which implies $F(Q) = 0$ which implies $F(f)$ is invertible
%   because exact functors preserve exact triangles.
% \end{proof}
% Given $F : A \to B$ between abelian categories,
% \begin{enumerate}
%   \item a right derived functor $RF$ should be
%   a \emph{left} Kan extension of $N_{dg}(\CH\,A) \to N_{dg}(\CH\,B) \to \cD(B)$ along
%   $N_{dg}(\CH\,A) \to \cD(A)$.
%   \item a left derived functor $LF$ should be
%   a \emph{right} Kan extension similarly.
% \end{enumerate}

% \cite[1.3.5.24]{lurie-HA} for comparison
% with animation.
Idea : The derived category $\cD(A)$ of an abelian category $A$ should 
have the following properties.
\begin{enumerate}
  \item $\cD(A)$ has zero object
  \item In $\cD(A)$, every morphism has kernel and cokernel.
  Things will get confusing so let us rename them to fiber and cofiber,
  reserving the words kernel and cokernel for abelian categories.
  A sequence $M \to N \to Q$ in $\cD(A)$ 
  is called left exact when it is fiber square.
  Similarly for right exact and cofiber sequence.
  \item In $\cD(A)$, left exact iff right exact iff exact.
  These are called exact triangles.
  \item Equipped with a fully faithful functor $A \to \cD(A)$
\end{enumerate}
All of above is satisfied by the category of chain complexes
with morphisms up to chain homotopy,
when suitably interpreted.
The following makes the difference and gives $\cD(A)$ : 
\begin{enumerate}[resume]
  \item A sequence $0 \to M \to N \to Q \to 0$ in $A$ should be
  short exact iff $M \to N \to Q$ is exact triangle in $\cD(A)$.\footnote{
    \textbf{This feels very much like 
  $C$ category with site structure,
  $\PSH C$ has all colimits,
  now quotient to $\SH C$ to force some coequalizers in $C$
  to be preserved under $C \to \PSH C$.
  Is there more to this?
  It would be nice to have a more canonical construction of
  $N_{dg}(\CH \, A)$ for Grothendieck abelian $A$.
  This is not quite animation because $A$ may not have enough
  compact projectives. 
  (E.g. category of abelian sheaves on a topological space.)
}
  }
\end{enumerate}
The properties (1) to (3) are word-for-word the definition
of a \emph{stable $\infty$-category},\cite[Def. 1.1.1.9]{lurie-HA}
modulo being precise about universal properties (zero object,
fiber, cofiber).
Property (4) will be the reason for \emph{$t$-structures}.
You are probably rolling your eyes: 
\emph{Why must $\infty$-categories be involved?}
Here are some reasons: 
\begin{enumerate}
  \item Abelian categories are too strict : 
  Property (3) forces all morphisms to be epimorphisms.
  Given a morphism $f : M \to N$, 
  we get a left exact sequence $\ker f \to M \to N$.
  If we have property (3) then we are forced $\mathrm{Im}\,f \simeq N$.
  The ``extra room'' granted by seeing things as 
  $\infty$-categories rather than 1-categories is 
  what will make property (3) possible.
  
  \item Triangulated categories are difficult to glue : 
  we will return to this point later.
  Let us say for now that 
  the theory of triangulated categories is literally a truncation
  of the theory of stable $\infty$-categories.\cite[Thm. 1.1.2.14]{lurie-HA}
\end{enumerate}
Here are some general intuition for thinking about $\infty$-categories : 
\begin{enumerate}
  \item There is really only one idea : 
  sets are seen as discrete spaces,
  then generalised to CW complexes \emph{up to homotopy}.
  The rest are consequences of this.
  ``CW complexes up to homotopy'' are what people
  mean by \emph{spaces}.
  \item Given an $\infty$-category $\cC$ and two objects $x , y$ in it,
  there is a space $\MAP_\bC(x , y)$ whose points are morphisms 
  $f : x \to y$ in $\cC$.
  \item Whenever you see ``there exists unique'' in 1-category theory,
  it is equivalent to asking for a set to be isomorphic to the singleton.
  In $\infty$-categories, you need to replace
  this with ``the space is contractible''.
  This underlies the entire theory of universal properties.
\end{enumerate}
Regarding derived functors : 
Given an additive functor $F : A \to B$,
there should be some induced functor $\cD(A) \to \cD(B)$.
When $F$ is say left exact, 
the induced functor should be left exact too.
But it turns out for functors between stable $\infty$-categories,
a functor is left exact iff right exact iff exact!
\cite[Prop. 1.1.4.1]{lurie-HA}
We will now make the following precise, but the answer to the first question is :
\begin{center}
  \text{Derived categories and derived functors are what you get
  when you force left exact iff right exact iff exact.}
\end{center}
Without further ado :
\begin{prop}

  Let $A$ be an additive category.
  Consider $\cK(A) \in \SSET$ 
  the dg nerve of the differential graded
  category of unbounded chain complexes in $A$.
  Then $\cK(A)$ is a stable infinity category.
  \textbf{TODO : Fill in details about standard $t$-structure
  giving $A$ as heart.}
\end{prop}
\begin{proof}
  \cite[Prop. 1.3.2.10]{lurie-HA}
\end{proof}
\cite[Remark 1.3.1.12]{lurie-HA}
shows that for $X , Y \in \cK(A)$,
we have an equivalence of spaces \[
  \MAP_{\cK(A)}(X , Y) \simeq \mathrm{DK}(\tau^{\leq 0} {\HOM}(X, Y))
\]
where $\HOM(X , Y) \in \CH\,\bZ$ is the hom complex,
$\tau^{\leq 0}$ is truncation killing all cohomologies in positive degree, 
and $\mathrm{DK}$ is the Dold-Kan equivalence
of 1-categories : \[
  \mathrm{DK} : \CH^{\leq 0}(\AB) \map{\sim}{} \FUN(\De^{\OP} , \AB)
\]
Applying \cite[Remark 1.2.3.14]{lurie-HA}, the homotopy groups
of the mapping space are the same as the cohomologies of the hom complex.
Since $\cK(A)$ is stable,
the mapping spaces naturally extend to mapping spectra.
We can access the negative homotopy groups by looping $Y$.
% \begin{dfn}
  
%   Let $\cC$ be an $\infty$-category and $W$ a collection of morphisms in $\cC$.
%   A functor $\cC \to \cC[W^{-1}]$ is called the localization
%   of $C$ at $W$ when for all $\infty$-categories $\cE$
%   restriction gives a fully faithful embedding \[
%     \FUN(\cC[W^{-1}] , \cE) \to \FUN(\cC , \cE)
%   \]
%   with essential image consisting of functors
%   which send morphisms in $W$ to isomorphisms.
% \end{dfn}
\begin{dfn}\cite[Definition 1.3.4.1]{lurie-HA}
  Let $A$ be an abelian category.
  Let $W$ denote the collection of quasi-isomorphisms in $\cK(A)$.
  We say an exact functor of stable $\infty$-categories 
  $L : \cK(A) \to \cD(A)$
  exhibits $\cD(A)$ as the derived stable $\infty$-category of $A$ when
  for any stable $\infty$-category $\cE$,
  restriction along $L$ induces a fully faithful embeding \[
    \FUN_{\mathrm{ex}}(\cD(A) , \cE) \to \FUN_{\mathrm{ex}}(\cK(A) , \cE)
  \]
  where $\FUN_{\mathrm{ex}}$ means the $\infty$-category of exact functors
  and we require the essential image to be precisely
  consisting of exact functors which invert $W$.
\end{dfn}
This is equivalent to the following formulation,
closer to the classical idea of Verdier quotients of triangulated categories : 
\begin{prop}
  An exact functor $L : \cK(A) \to \cD(A)$ of stable $\infty$-categories
  exhibits $\cD(A)$ as the derived stable $\infty$-category of $A$
  iff for all stable infinity categories $\cE$,
  restriction along $L$ induces 
  a fully faithful embedding of infinity categories \[
    \FUN_{\mathrm{ex}}(\cD(A), \cE) \map{}{}
    \FUN_{\mathrm{ex}}(\cK(A) , \cE)
  \]
  with essential image consisting precisely of 
  functors where every acyclic $X$ maps to a zero object.
\end{prop}
\begin{proof}
  Let $F : \cK(A) \to \cE$ be an exact functor.
  If $F$ inverts quasi-isomorphisms,
  then $F(Q) = 0$ for any acyclic $Q$ because $0 \to Q \in W$.
  Conversely, if $F$ kills acyclics,
  then $f \in W$ implies its cofiber $Q$ is acyclic by long exact
  sequence of cohomologies associated to the standard $t$-structure 
  (or just direct computation),
  which implies $F(Q) = 0$ which implies $F(f)$ is invertible
  because exact functors preserve exact triangles.
\end{proof}
Let $F : A \to B$ an additive functor of abelian categories.
Suppose we have derived categories for $A$ and $B$.
The functor $F$ induces an exact functor $\cK(A) \to \cK(B)$.
\begin{center}
  \text{Question : Can we extend $F$ along derived categories? }
\end{center}
\begin{cd}
  {\mathcal{K}(A)} & {\mathcal{K}(B)} \\
	{\mathcal{D}(A)} & {\mathcal{D}(B)}
	\arrow["F", from=1-1, to=1-2]
	\arrow[from=1-1, to=2-1]
	\arrow[from=1-2, to=2-2]
	\arrow["{?}", dashed, from=2-1, to=2-2]
\end{cd}
In general, this won't be possible because
$F$ need not take quasi-isomorphisms to quasi-isomorphisms.
\begin{center}
  \text{Idea : approximate $X \in \cD(A)$ with objects from $\cK(A)$.} 
\end{center}
There are two directions we can do : look at morphisms $LY \to X$
or look at morphisms $X \to LY$.
Let us focus on the morphisms from the left.
More precisely, consider the comma category $\cK(A)_{/ X}$
whose objects are pairs $(Y , L Y \to X)$ where $Y \in \cK(A)$
and morphisms are ones in $\cK(A)$ commuting with
the map to $X$ after applying $L$.
Consider the diagram in $\cD(B)$ : 
\[
  \cK(A)_{/X} \to \cK(A) \map{F}{} \cK(B) \to \cD(B)
\]
Then the ``best approximation of $F$ at $X$ from the left'' is defined as
\[
  \text{ ``$RF(X) := \COLIM_{(Y \in \cK(A) , LY \to X)} F(Y) \in \cD(B)$''}
\]
Doing this at every $X \in \cD(A)$, 
we get the so-called \emph{(pointwise) 
  left Kan extension of $F$ along $\cK(A) \to \cD(A)$}.\cite[
    \href{https://kerodon.net/tag/0300}{Tag 0300}
    ]{kerodon}
However, I put quotation marks because this runs into two issues : 
\begin{enumerate}
  \item The colimit may not exist in $\cD(B)$
  \item If $RF$ is defined
  on terms of an exact triangle $M \to N \to Q$,
  we would like $RF$ to take this to an exact triangle in $\cD(B)$,
  i.e. we want $RF$ to be an exact functor.
  This is not immediately clear from the formula.
\end{enumerate}
Stable $\infty$-categories like $\cD(B)$ are not far off
from having all colimits; they already have finite colimits.
For $\infty$-categories with finite colimits,
there is a canonical way to formally add in all colimits
whilst preserving the finite colimits already present : 
the ind-completion.
We proceed as follows : 
\begin{dfn}
  Consider the commutative diagram : 
  \begin{cd}
    {\cK(A)} & {\cK(A)} & {\mathcal{D}(B)} \\
    {\mathcal{D}(A)} & {\mathrm{Ind}\,\cK(A)} & {\mathrm{Ind}\,\mathcal{D}(B)}
    \arrow[from=1-1, to=2-1]
    \arrow["{=}"', from=1-2, to=1-1]
    \arrow["F", from=1-2, to=1-3]
    \arrow[from=1-2, to=2-2]
    \arrow[from=1-3, to=2-3]
    \arrow["{\mathrm{Lan}}"', from=2-1, to=2-2]
    \arrow["{\mathrm{Ind}\,F}"', from=2-2, to=2-3]
  \end{cd}
  where \begin{enumerate}
    \item each Yoneda embedding $\cK \to \PSH \cK$
    factors through the inclusion $\IND \cK \subs \PSH \cK$
    and preserves finite colimits.
    \item The ind-completions are stable. \cite[1.1.3.6]{lurie-HA}
    \item The ind-completions have arbitrary colimits
    because $\cK(A) , \cD(B)$ have finite colimits.
    \item $\cD(A) \to \IND\,\cK(A)$ is the restriction of
    the Yoenda embedding.
    On objects, it takes $X \in \cD(A)$
    to $\HOM(L \,\_ , X)$.
    Under the equivalence of presheaves of spaces
    and right fibrations,
    $\HOM(L \,\_ , X)$ corresponds to
    precisely the diagram $\cK(A)_{/ X} \to \cK(A)$.
    Since $\IND \subs \PSH$ is stable under finite limits,
    we have that $\cD(A) \to \IND\,\cK(A)$ is exact.
  \end{enumerate}
  The composition of bottom row is thus an exact functor,
  which we call the \emph{right derived functor $RF$ of $F$}.
  We say $RF$ is defined at $X \in \cD(A)$ when
  $RF(X)$ is in the essential image of $\cD(B)$.
  For $X \in \cK(A)$,
  it is customary to write $RF(X)$ for $RF(LX)$.
\end{dfn}
There are still several problems : 
\begin{itemize}
  \item (Q1) How can we compute $RF$?
  \item (Q2) If $A$ is not small, like the 1-category of abelian groups,
  then the homs of, say, the classical derived stable $\infty$-category $h \cD(A)$
  are too large to be sets!
\end{itemize}
For (Q1), we seek to find conditions so that
the colimit diagram simplifies and lands in $\cD(B)$.
The most ideal situation would be if we can replace the diagram 
by constant diagram without changing the colimit.
This happens iff the diagram has a final object,
\cite[\href{https://kerodon.net/tag/03LQ}{Tag 03LQ}]{kerodon}
so this is what we ask for!
\begin{prop}\label{derived:cofinal}
  
  Let $X \in \cD(A)$. Suppose the functor 
  $\HOM_{\cD(A)}(L \_ , X)$ on $\cK(A)$ is representable
  by some $I \in \cK(A)$, i.e. we have an isomorphism \[
    \HOM_{\cD(A)}(L \_ , X) \map{\sim}{} \HOM_{\cK(A)}(\_ , I)
  \]
  In particular, the identity of $I$ gives 
  a point \[
    \De^0 \to \cK(A)_{/ X}
  \]
  This inclusion is right cofinal.
  Hence the right derived functor of $F$ is defined at $X$ 
  and we have $RF(X) \simeq F(I)$.
\end{prop}
\begin{proof}
  Under the equivalence of contravariant functors into $\mathrm{Spaces}$
  and right fibrations,
  the assumption is precisely an isomorphism 
  between right fibrations \begin{cd}
    {\mathcal{K}(A)_{/ X}} && {\mathcal{K}(A)_{/ I}} \\
    & {\mathcal{K}(A)}
    \arrow["\sim"', from=1-1, to=1-3]
    \arrow[from=1-1, to=2-2]
    \arrow[from=1-3, to=2-2]
  \end{cd}
  By preservation of colimits 
  under changing the diagram up to categorical equivalence,
  \cite[\href{https://kerodon.net/tag/02N5}{Tag 02N5}]{kerodon}
  we now seek the colimit of the diagram 
  $\cK(A)_{/ I} \to \cK(A) \to \cD(B) \to \IND\,\cD(B)$.
  We have the final object in $\cK(A) / I$ given by $\id : I \to I$.
  The inclusion of a point is right cofinal iff the point is a final object.
  \cite[\href{https://kerodon.net/tag/03LQ}{Tag 03LQ}]{kerodon}
  Thus the desired colimit is computed as 
  the image $F(I)$.
  \cite[\href{https://kerodon.net/tag/02XW}{Tag 02XW}]{kerodon}
\end{proof}
\begin{center}
  Punchline : to compute $RF$ on all of $\cD(A)$,
  it suffices to find a right adjoint to the localization functor 
  $L : \cK(A) \to \cD(A)$.\footnote{
    Dualize \cite[\href{https://kerodon.net/tag/02FV}{Tag 02FV}]{kerodon}
    for precise statment about existence of right adjoint
    from pointwise representability.
  }
\end{center}
Of course, the dual story for \emph{the left derived functor of $F$}
is true by reversing all arrows, 
replacing colimits with limits,
ind-completion with pro-completion. 
(Keeping in mind the dual of 
a stable $\infty$-category is again stable!)

It turns out there is a situation where one can find a 
\emph{fully faithful} right adjoint to $L$,
realising $\cD(A)$ as a \emph{full subcategory of $\cK(A)$}.
This general pattern of finding localizations as full subcategories
is called \emph{reflective localization}, or 
in model category theory called \emph{Bousfield localization}.
The dual notion when one can find a full faithful \emph{left} adjoint
to the localization functor $L$ is called \emph{coreflective localization}.
\begin{dfn}
  \cite[\href{https://kerodon.net/tag/02G0}{Tag 02G0}]{kerodon}

  Let $\cC$ be an $\infty$-category and $W$ a collection of morphisms.
  $W$ is called \emph{localizing} when the following are satisfied : 
  \begin{enumerate}
    \item isomorphisms are in $W$
    \item For any commuting triangle in $C$
    \begin{cd}
      X && Z \\
      & Y
      \arrow["c", from=1-1, to=1-3]
      \arrow["a"', from=1-1, to=2-2]
      \arrow["b"', from=2-2, to=1-3]
    \end{cd}
    with $b \in W$, then $a \in W$ iff $c \in W$.
    \item For all $X \in \cC$ there exists 
    a morphism $X \to \TILDE{X}$ in $W$ such that $\TILDE{X}$ is
    $W$-local, meaning for all $f \in W$, $\MAP_C(f , \TILDE{X})$
    is an equivalence of spaces.
  \end{enumerate}
  Dually, $W$ is called \emph{co-localizing} when it satisfies
  (1) and (2) with a dual version of (3) : 
  For all $X \in \cC$ there exists $\TILDE{X} \to X$ in $W$
  with $\TILDE{X}$ $W$-co-local, meaning
  for all $f \in W$, $\MAP_\cC(\TILDE{X} , f)$ is an equivalence of spaces.
\end{dfn}
The significance of localizing / co-localizing collections of morphisms 
is the following : 
\begin{prop}[Characterisation of (co)reflective localizations]
  \cite[\href{https://kerodon.net/tag/02G3}{Tag 02G3}]{kerodon}
  \label{local:fundamental}
  Let $\cC$ be an $\infty$-category
  and $W$ a collection of morphisms.
  Let $\cI \subs \cC$ is the full subcategory of $W$-local objects.
  Then TFAE : 
  \begin{enumerate}
    \item $W$ is localizing.
    \item $\cI$ is reflective and $W$ is precisely
    the collection of morphisms $f$ such that $L(f)$.
  \end{enumerate}
  In this case, $\cI \subs \cC$ has a left adjoint $L$
  and $L$ exhibits $\cI$ as the localization of $\cC$ at $W$.

  We have dual statements for when $W$ is co-localizing.
\end{prop}
\begin{proof}
  (1) iff (2) is the equivalence of (a) and (c)
  in \cite[\href{https://kerodon.net/tag/05Z2}{Tag 05Z2}]{kerodon}.

  A full subcategory of reflective iff the inclusion 
  admits a left adjoint.\cite[\href{https://kerodon.net/tag/02FA}{Tag 02FA}]{kerodon}

  $L$ is then a localization functor.
  \cite[\href{https://kerodon.net/tag/04JL}{Tag 04JL}]{kerodon}
  $L$ exhibits $\cI$ as the localization of $\cC$ at
  the collection of morphisms $f$ such that $L(f)$.
  \cite[\href{https://kerodon.net/tag/04JH}{Tag 04JH}]{kerodon}
  This is precisely $W$ by (2).
\end{proof}

We will prove that when $\cC = \cK(A)$ for abelian $A$ 
and $W$ the collection of quasi-isomorphisms,
then certain conditions make $W$ localizing or co-localizing.
Some terminology : in the case of $\cK(A)$,
$W$-local objects are called $K$-injectives 
and $W$-co-local objects are called $K$-projectives.
\cite[\href{https://stacks.math.columbia.edu/tag/070G}{Tag 070G}]{stacks-project}
The are also respectively called homotopy injectives,
homotopy projectives in \cite[Section 14.3]{KS06}.

\begin{prop}[Derived categories via reflective localization]
  Let $A$ be Grothendieck abelian,
  i.e. presentable abelian category with exact filtered colimits.
  Then the collection $W$ of quasi-isomorphisms is localizing.
  Hence \begin{enumerate}
    \item $\cI$ the full subcategory of
    $K$-injective objects is stable.
    \item The inclusion $\cI \subs \cK(A)$ has a left adjoint
    $L$, and it exhibits $\cI$ as a derived stable $\infty$-category of $A$.
  \end{enumerate}
\end{prop}
\begin{proof}
  The first two conditions for $W$ being localizing is clear.
  We need to produce for each $X \in \cK(A)$
  a quasi-isomorphism $X \to I$ such that $I$ is $K$-injective.
  % And now enters the non-formal input :
  % there is a left proper combinatorial model structure on $\CH\,A$.
  % \cite[Lem. 1.3.5.3]{lurie-HA}
  % It follows that every $X \in \cK(A)$ admits
  % a trivial cofibration $X \to I$ with $I$ fibrant.
  % \cite[Lem. 1.3.5.14]{lurie-HA} shows that
  % fibrant $I$ lie in $\cI$,
  % completing the proof that $W$ is localizing.
  We refer the reader to \cite[Theorem 14.3.1]{KS06}.
  It is clear that $\cI$ is closed under translation
  and cofibers, and hence (1).
  \ref{local:fundamental} gives (2).
\end{proof}
The size issue (Q2) is solved if when one can show
the homs in $h \cI$ are small.
This is covered in \cite[Prop. 1.3.5.14]{lurie-HA}.
Let us note that we can also have co-localizing situation.
\begin{prop}[Derived categories via coreflective localization]
  Let $A$ be abelian with small direct sums,
  which are also exact. (In particular, $A$ Grothendieck satisfies this.)
  Suppose $A$ has enough projectives.
  Then the collection $W$ of quasi-isomorphisms is co-localizing.
  Hence \begin{enumerate}
    \item the full subcategory $\cP$ of $K$-projectives is stable.
    \item the inclusion $\cP \subs \cK(A)$ has a right adjoint $L$,
    and it exhibits $\cP$ as a derived stable $\infty$-category of $A$.
  \end{enumerate}
\end{prop}
\begin{proof}
  Again, first two conditions of $W$ being co-localizing is clear.
  \cite[Theorem 14.4.3]{KS06} produces for each $X \in \cK(A)$
  a quasi-isomorphism $P \to X$ with $P \in \cP$,
  proving $W$ is co-localizing.
  (1) follows because again $P$
\end{proof}
Now, given Grothendieck abelian $A$ with enough projectives,
we can always take left and right derived functors!
In this case, we have the following adjunction for any
stable $\infty$-category $\cE$ : 
\begin{cd}
  {\mathcal{P}} \\
	{\mathcal{D}(A)} & {\mathcal{K}(A)} & \rightsquigarrow & {\mathrm{Fun}_{\mathrm{ex}}(\mathcal{K}(A) , \mathcal{E})} & {\mathrm{Fun}_{\mathrm{ex}}(\mathcal{D}(A) , \mathcal{E})} \\
	{\mathcal{I}}
	\arrow["\sim"', from=1-1, to=2-1]
	\arrow["p", shift left=2, from=1-1, to=2-2]
	\arrow["\bot"{marking, allow upside down}, draw=none, from=1-1, to=2-2]
	\arrow[shift left=2, from=2-2, to=1-1]
	\arrow["\pi"{description}, from=2-2, to=2-1]
	\arrow[shift right=2, from=2-2, to=3-1]
	\arrow["{F \mapsto RF = F i}", shift left=5, from=2-4, to=2-5]
	\arrow["{F \mapsto LF = F p}"', shift right=5, from=2-4, to=2-5]
	\arrow["\bot"{description}, shift left=3, draw=none, from=2-4, to=2-5]
	\arrow["\bot"{description}, shift right=3, draw=none, from=2-4, to=2-5]
	\arrow["{\pi^*}"{description}, from=2-5, to=2-4]
	\arrow["\sim", from=3-1, to=2-1]
	\arrow["i"', shift right=2, from=3-1, to=2-2]
	\arrow["\bot"{marking, allow upside down}, draw=none, from=3-1, to=2-2]
\end{cd}
where \begin{enumerate}
  \item we have \emph{two} incarnations of the derived stable $\infty$-category
  $\pi : \cK(A) \to \cD(A)$.
  Notice that $\cP$ and $\cI$ are \emph{not} equal as 
  full subcategories of $\cK$.
  \item taking \emph{left} derived functors by composing with $K$-projective
  replacement gives \emph{right} adjoint to restricting functors along $\pi$,
  A.K.A. (pointwise) \emph{right} Kan extensions. 
  It is unfortunate that the left/right naming does not match.
  \item taking \emph{right} derived functors by composing with $K$-injective
  replacement gives a \emph{left} adjoint to restricting functors along $\pi$,
  A.K.A. (pointwise) \emph{left} Kan extension.
\end{enumerate}
It is possible to be in the Grothendieck abelian case,
with a priori not enough projectives.
However, if you happen to have specific bounded above
projective complexes, then you can still compute some values of 
left derived functors.
\begin{prop}
  
  Let $F : A \to B$ be additive functor between abelian categories
  with derived stable $\infty$-categories.
  Suppose $W$ is localizing for $\cK(A)$ but assume $F$ is \emph{right exact}.
  Let $X \in \cD(A)$ and $P \to X$ a quasi-isomorphism
  with a cohomologically bounded above complex of projective objects in $A$.
  Then $LF(X)$ is defined, and computes $F(P)$.
\end{prop}
\begin{proof}
  \cite[Lem. 1.3.2.20]{lurie-HA} shows that $P$ is $K$-projective.
  Then \[
    \HOM(P , \_) \map{\sim}{} \HOM(P , L \_) \map{\sim}{} \HOM(X , L \_)
  \]
  where $L : \cK(A) \to \cI \simeq \cD(A)$ is the localization functor,
  the first equivalence is $P$ being $K$-projective,
  and the second is $L(\_)$ being $K$-injective.
  The result follows from the dual of \ref{derived:cofinal}.
\end{proof}

One can ask for the comparison of $RF$ to $F$ : 
\begin{prop}
  Let $F : A \to B$ an additive functor of abelian categories.
  Assume that $W$ in $\cK(A)$ is localizing.
  Suppose we also have a derived stable $\infty$-category of $B$.
  Assume $F$ is left exact.
  Then there is an induced isomorphism $H^0(RF(X)) \to F(X)$.
  \footnote{Converse is probably true, but I am uninterested.}

  Similarly for left derived functors and enough projectives.
\end{prop}
\begin{proof}
  Let $I$ be the image of $X$ in $\cD(A)$
  and $\eta : X \to I$ the unit morphism of the adjunction in $\cK(A)$.
  Because $\cD(A) \to \cK(A)$ is fully faithful right adjoint,
  $L(\eta)$ is an isomorphism and hence $\eta$ is a quasi-isomorphism.
  \ref{local:fundamental}
  Since $H^i X = X$ when $i = 0$ and zero otherwise,
  we can replace $I$ with a complex concentrated in cohomological degrees 
  $\geq 0$, exact everywhere except $X = H^i(X) \simeq H^0(I)$.
  Then \[
    RF(X) \simeq F(I)
  \]
  as objects in $\cD(B)$.
  Then $H^0(RF(X)) \simeq H^0(F(I)) = \mathrm{Ker}(F I^0 \to F I^1)$.
  Now use $F$ left exact.
\end{proof}
\begin{center}
  Question : These $K$-injective / $K$-projective resolutions
  seem quite opaque. Can we compute derived functors using 
  more efficient resolutions?
\end{center}
To answer this, we introduce \emph{acyclic resolutions}.
This requires $W$ to be co-localizing to the left derived functors
and localizing to do right derived functors.
\begin{prop}\label{derived:acyclic_res}
  Let $F : A \to B$ a left exact additive functor between abelian categories.
  Assume we have derived stable $\infty$-categories.
  Further assume for $A$ that $W$ is co-localizing
  so $LF$ is defined on all of $\cD(A)$.
  Now let $X \in \cK(A)$ and $P \to X$ be a quasi-isomorphism
  from a complex of \emph{acyclics for $LF$}, meaning 
  for $n \in \bZ$, we have $LF(P_n) \to P_n$ is an isomorphism in $\cD(B)$.
  Then $LF(X) \simeq F(P)$.
  We have dual results for right derived functors.
\end{prop}
\begin{proof}
  % \cite[\href{https://stacks.math.columbia.edu/tag/015E}{Tag 015E}]{stacks-project}
  Let $\TILDE{P}$ be the image of $X$ in $\cP$.
  Then by the $\infty$-categorical adjunction
  and two-out-of-three for quasi-isomorphisms,
  we have a quasi-isomorphism $f : \TILDE{P} \to P$.
  This gives an exact triangle $F(\TILDE{P}) \to F(P) \to F(\mathrm{Cone}\,f)$.
  By the explicit construction of $\mathrm{Cone}\,f$ and
  the fact that both $\TILDE{P}$ and $P$ are acyclic for $LF$,
  we have that $\mathrm{Cone}\,f$ is also acyclic for $LF$.
  By $f$ being quasi-isomorphism implies $\mathrm{Cone}\,f$ is acyclic.
  It is an easy exercise to show that 
  acyclicity of $\mathrm{Cone}\,f$ for $LF$ now implies
  $F(\mathrm{Cone}\,f)$ is acyclic.
  Since this is the cofiber for $F(\TILDE{P}) \to F(P)$,
  we deduce that $LF(X) \simeq F(\TILDE{P}) \simeq F(P)$ in $\bD(B)$.
\end{proof}
The above implies, for example,
one can compute $\_ \otimes_A^L \_$ in $\cD(\MOD_A)$ for a commutative ring $A$
using flat modules.
We are also free to cook up any resolution as long as
it is acyclic for the derived functor.

\textbf{TODO : If $W$ is co-localizing for $\cK(A)$
and $A$ has $\_ \otimes \_$ lifting to $\mathrm{Tot}(\_ \otimes \_)$
on $\cK(A)$,
then showing say $\cP$ is closed under $\mathrm{Tot(\_ \otimes \_)}$
should give symmetric monoidal structure on $\cD(A)$?}

\section{The stable $\infty$-category of quasi-coherent sheaves for a scheme}

\textbf{TODO}
\begin{enumerate}
  \item A scheme $X$ is a functor which is a small colimit of affines.
  In fact, determined by right cofinal subdiagram of the affine opens.
  \item Define $\QCOH\,X := \LIM_{\SPEC A \to X} D(\MOD_A)$.
  Works because small limits of stable $\infty$-categories is
  again a stable $\infty$-category.
  This is easier than triangulated categories 
  because stability is a \emph{property}
  whilst a triangulated structure on an additive category is \emph{data}.
  \item $\QCOH\,X \simeq \LIM_{U \text{ affine open }\subs X} D(\MOD_{\cO(U)})$
  and Zariski localization is flat so the $t$-structure 
  on affine opens define a $t$-structure on $\QCOH\,X$.
  Specifically, $(\QCOH\,X)^{\leq 0} := \LIM_{U} D^{\leq 0}(\MOD_{\cO(U)})$.
  \item Prove $(\QCOH \,X)^\heartsuit \simeq \LIM_{U} \MOD_{\cO(U)}$
  which is the usual definition of quasi-coherent sheaves.
\end{enumerate}

\section{Cech cohomology and more generally flat base change}

\textbf{TODO}
\begin{enumerate}
  \item Affines having no higher cohomology
  means we have ``a good approximation of a scheme
  by cohomological simplicies'' in the same sense
  as if we built a CW complex by gluing simplicies.
  Explain how taking tubular neighbourhoods of faces gives
  a Cech cover of CW complex.
  \item Cech complex being resolution is just Zariski descent.
  Previous point implies Cech complex is
  acyclic resolution for derived direct image.
  Hence Cech cohomology computes higher direct images by 
  \ref{derived:acyclic_res}.
  \item Flat base change for qcqs morphisms is just realising
  Cech complex only requires interesction of affines to be 
  covered by finitely many affines and computes derived direct image
  for base change because of flatness assumption.
\end{enumerate}

\printbibliography

\end{document}